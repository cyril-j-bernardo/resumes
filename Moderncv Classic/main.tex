%% start of file `template.tex'.
%% Copyright 2006-2013 Xavier Danaux (xdanaux@gmail.com).
%
% This work may be distributed and/or modified under the
% conditions of the LaTeX Project Public License version 1.3c,
% available at http://www.latex-project.org/lppl/.


\documentclass[11pt,a4paper,sans]{moderncv}        % possible options include font size ('10pt', '11pt' and '12pt'), paper size ('a4paper', 'letterpaper', 'a5paper', 'legalpaper', 'executivepaper' and 'landscape') and font family ('sans' and 'roman')

% moderncv themes
\moderncvstyle{classic}                            % style options are 'casual' (default), 'classic', 'oldstyle' and 'banking'
\moderncvcolor{green}                              % color options 'blue' (default), 'orange', 'green', 'red', 'purple', 'grey' and 'black'
%\renewcommand{\familydefault}{\sfdefault}         % to set the default font; use '\sfdefault' for the default sans serif font, '\rmdefault' for the default roman one, or any tex font name
%\nopagenumbers{}                                  % uncomment to suppress automatic page numbering for CVs longer than one page

% character encoding
\usepackage[utf8]{inputenc}                       % if you are not using xelatex ou lualatex, replace by the encoding you are using
%\usepackage{CJKutf8}                              % if you need to use CJK to typeset your resume in Chinese, Japanese or Korean

% adjust the page margins
\usepackage[scale=0.75]{geometry}
%\setlength{\hintscolumnwidth}{3cm}                % if you want to change the width of the column with the dates
%\setlength{\makecvtitlenamewidth}{10cm}           % for the 'classic' style, if you want to force the width allocated to your name and avoid line breaks. be careful though, the length is normally calculated to avoid any overlap with your personal info; use this at your own typographical risks...
\newcommand{\CompanyName}{General Electric}
\newcommand{\ProgramName}{Edison Engineering Development Program}
\newcommand{\PositionName}{Photonics}

% personal data
\name{Cyril Jerome}{Bernardo}
%\title{Resumé title}                               % optional, remove / comment the line if not wanted
%\address{street and number}{postcode city}{country}% optional, remove / comment the line if not wanted; the "postcode city" and and "country" arguments can be omitted or provided empty
\phone[mobile]{+1~(718)~704~7653}                   % optional, remove / comment the line if not wanted
%\phone[fixed]{+2~(345)~678~901}                    % optional, remove / comment the line if not wanted
%\phone[fax]{+3~(456)~789~012}                      % optional, remove / comment the line if not wanted
\email{cyril.j.bernardo@gmail.com}                               % optional, remove / comment the line if not wanted
\homepage{www.linkedin.com/in/cyrilbernardo}                         % optional, remove / comment the line if not wanted
\extrainfo{MS Mechanical Engineering - May 2017}                 % optional, remove / comment the line if not wanted
\photo[64pt][0.4pt]{picture}                       % optional, remove / comment the line if not wanted; '64pt' is the height the picture must be resized to, 0.4pt is the thickness of the frame around it (put it to 0pt for no frame) and 'picture' is the name of the picture file
\quote{Some quote}                                 % optional, remove / comment the line if not wanted

% to show numerical labels in the bibliography (default is to show no labels); only useful if you make citations in your resume
%\makeatletter
%\renewcommand*{\bibliographyitemlabel}{\@biblabel{\arabic{enumiv}}}
%\makeatother
%\renewcommand*{\bibliographyitemlabel}{[\arabic{enumiv}]}% CONSIDER REPLACING THE ABOVE BY THIS

% bibliography with mutiple entries
%\usepackage{multibib}
%\newcites{book,misc}{{Books},{Others}}
%----------------------------------------------------------------------------------
%            content
%----------------------------------------------------------------------------------
\begin{document}
%-----       letter       ---------------------------------------------------------
% recipient data
\recipient{\CompanyName{}}{}
\date{\today}
\opening{Dear Sir or Madam,}
\closing{Yours faithfully,}
\makelettertitle

I am Cyril Bernardo, a MS Mechanical Engineering student at New York University. Taking part in \CompanyName{}'s \ProgramName{} for \PositionName{} would be the next step in my career as an aspiring engineer. The \ProgramName{} advertises an eclectic program with rotations in a wide range of places; this is perfect for me as someone who has a diverse background in govenment research, laboratory research, technology, and leadership.

In the past, I have worked with developing tools that were dependent on optoelectronic systems. For a Mechatronics class, I created a solution to prevent vehicular heat stroke in pets and children by creating a system that would actuate based on stimuli to infrared sensors. This past summer, I developed a computer vision tool using live feed data from off-the-shelf webcams at Warby Parker. In my current internship at the NASA Johnson Space Center, I am processing acoustic data as well as developing multi-physics analyses in COMSOL. Although wide-in-breadth, all experiences combined has shown me what a broad field engineering is--and I've only scratched the surface.

The \ProgramName{} is one that boasts of the opportunity to work cross-functionally in diverse teams with an added benefit of furthering the technical exposure of the participant. If there is anything that my time in university has taught me, it is to always take the chance to do something different and to do something meaningful. I look forward to speaking with you to give and gain insight on my experience and the program.


\makeletterclosing

\end{document}


%% end of file `template.tex'.
